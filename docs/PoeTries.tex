\documentclass[]{book}
\usepackage{lmodern}
\usepackage{amssymb,amsmath}
\usepackage{ifxetex,ifluatex}
\usepackage{fixltx2e} % provides \textsubscript
\ifnum 0\ifxetex 1\fi\ifluatex 1\fi=0 % if pdftex
  \usepackage[T1]{fontenc}
  \usepackage[utf8]{inputenc}
\else % if luatex or xelatex
  \ifxetex
    \usepackage{mathspec}
  \else
    \usepackage{fontspec}
  \fi
  \defaultfontfeatures{Ligatures=TeX,Scale=MatchLowercase}
\fi
% use upquote if available, for straight quotes in verbatim environments
\IfFileExists{upquote.sty}{\usepackage{upquote}}{}
% use microtype if available
\IfFileExists{microtype.sty}{%
\usepackage{microtype}
\UseMicrotypeSet[protrusion]{basicmath} % disable protrusion for tt fonts
}{}
\usepackage[margin=1in]{geometry}
\usepackage{hyperref}
\hypersetup{unicode=true,
            pdftitle={PoeTries: I tentativi di Poe},
            pdfauthor={Alan Poe},
            pdfborder={0 0 0},
            breaklinks=true}
\urlstyle{same}  % don't use monospace font for urls
\usepackage{natbib}
\bibliographystyle{apalike}
\usepackage{longtable,booktabs}
\usepackage{graphicx,grffile}
\makeatletter
\def\maxwidth{\ifdim\Gin@nat@width>\linewidth\linewidth\else\Gin@nat@width\fi}
\def\maxheight{\ifdim\Gin@nat@height>\textheight\textheight\else\Gin@nat@height\fi}
\makeatother
% Scale images if necessary, so that they will not overflow the page
% margins by default, and it is still possible to overwrite the defaults
% using explicit options in \includegraphics[width, height, ...]{}
\setkeys{Gin}{width=\maxwidth,height=\maxheight,keepaspectratio}
\IfFileExists{parskip.sty}{%
\usepackage{parskip}
}{% else
\setlength{\parindent}{0pt}
\setlength{\parskip}{6pt plus 2pt minus 1pt}
}
\setlength{\emergencystretch}{3em}  % prevent overfull lines
\providecommand{\tightlist}{%
  \setlength{\itemsep}{0pt}\setlength{\parskip}{0pt}}
\setcounter{secnumdepth}{5}
% Redefines (sub)paragraphs to behave more like sections
\ifx\paragraph\undefined\else
\let\oldparagraph\paragraph
\renewcommand{\paragraph}[1]{\oldparagraph{#1}\mbox{}}
\fi
\ifx\subparagraph\undefined\else
\let\oldsubparagraph\subparagraph
\renewcommand{\subparagraph}[1]{\oldsubparagraph{#1}\mbox{}}
\fi

%%% Use protect on footnotes to avoid problems with footnotes in titles
\let\rmarkdownfootnote\footnote%
\def\footnote{\protect\rmarkdownfootnote}

%%% Change title format to be more compact
\usepackage{titling}

% Create subtitle command for use in maketitle
\newcommand{\subtitle}[1]{
  \posttitle{
    \begin{center}\large#1\end{center}
    }
}

\setlength{\droptitle}{-2em}
  \title{PoeTries: I tentativi di Poe}
  \pretitle{\vspace{\droptitle}\centering\huge}
  \posttitle{\par}
  \author{Alan Poe}
  \preauthor{\centering\large\emph}
  \postauthor{\par}
  \predate{\centering\large\emph}
  \postdate{\par}
  \date{2017-10-08}

\usepackage{booktabs}
\usepackage{amsthm}
\makeatletter
\def\thm@space@setup{%
  \thm@preskip=8pt plus 2pt minus 4pt
  \thm@postskip=\thm@preskip
}
\makeatother

\begin{document}
\maketitle

{
\setcounter{tocdepth}{1}
\tableofcontents
}
\chapter*{Making poetries is the new making
code!}\label{making-poetries-is-the-new-making-code}
\addcontentsline{toc}{chapter}{Making poetries is the new making code!}

This is the first book of poetries here on \textbf{GitHub} written in
\textbf{Markdown}.

Please take a breath from coding, relax and read some poetries. Making
poetries is the new making code! You need to be: - essential - versatile
- elegant

\section*{Here it's me}\label{here-its-me}
\addcontentsline{toc}{section}{Here it's me}

My name is rabolas, but people call me rabo.

Every bio - as it should be - containsa series of personal information.
In preserving this dubious convention, here are mine:

\begin{itemize}
\tightlist
\item
  I walked through endless stretches of bluebells blooming in
  Oxfordshire and with ducks along the Thames. Maybe it's trivial but
  for me it's a good memory
\item
  I was in Kensington, not London but Cape Town, where white people are
  0.3\% of the population. I felt observed, but I was in good company
\item
  I like numbers. Isn't funny for someone who thinks to be a poet and a
  writer?
\item
  At a park in Montreal a squirrel stole my lunch
\item
  I'm used to read newspapers article from the end to the beginning
\item
  I like listening people and I like hiding myself into the things I
  write
\end{itemize}

\section*{Support or Contact}\label{support-or-contact}
\addcontentsline{toc}{section}{Support or Contact}

Having troubles with all this mess? Check out my blog
\href{http://litteratti.wordpress.com/}{Litteratti} or {[}contact me{]}
at rabolas at gmail.com and I will help you sort it out.

\chapter{Reflections}\label{reflections}

Shattered mirror reflections

seeded down the street along the night.

Tall trees branches bare

beneath my feet run clouds.

And like a flower that opens the night

a swatch of sky in a ray of sunshine:

a sketch of light melts down the street

turns into a shadow, a Flash

of elsewhere. And I, who are just a reflection

in this nothing copyright, another

myself. An image skewed.

A pebble thrown among puddles scattered

my circulars horizons.

\section*{Riflessioni (the Italian original
version)}\label{riflessioni-the-italian-original-version}
\addcontentsline{toc}{section}{Riflessioni (the Italian original
version)}

Frantumi di specchio seminati

per strada lungo la notte.

Alberi altissimi dai rami scarni

sotto i miei piedi corrono nuvole.

E come un fiore che si apre la notte

un ritaglio di cielo in un raggio di sole:

uno schizzo di luce si scioglie per strada

si trasforma in un'ombra, uno sprazzo

d'altrove. E io, che sono solo un riflesso

in questo nulla d'autore, un altro

me stesso. Un'immagine obliqua.

Un sassolino gettato tra pozzanghere sparse

i miei circolari orizzonti.


\end{document}
